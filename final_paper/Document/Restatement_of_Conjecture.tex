
\section{Restating the Conjecture with Mathematical Precision}

The conjecture posits that for the binary expansion of \(2\), the length of any run of consecutive zeros starting at position \(n\) is bounded by a logarithmic function of \(n\). This can be written formally as:
\[
k_n \leq \log_2(n) + C,
\]
where:
\begin{itemize}
    \item \(k_n\) is the length of the run of zeros starting at position \(n\),
    \item \(C\) is a constant, which can depend on the properties of \(2\), but is expected to be independent of \(n\).
\end{itemize}
The inequality must hold for all sufficiently large \(n\).

\section*{Additional Conditions and Assumptions}

To ensure clarity and rigor, the following conditions and assumptions are introduced:

\subsection*{Binary Expansion Definition}

The binary expansion of \(2\) is treated as an infinite sequence of bits, \(\{b_1, b_2, b_3, \ldots\}\), where \(b_i \in \{0, 1\}\). A run of \(k_n\) zeros at position \(n\) implies \(b_n = b_{n+1} = \ldots = b_{n+k_n-1} = 0\), and \(b_{n+k_n} = 1\).

\subsection*{Connection to Approximation}

A run of \(k_n\) zeros implies that the binary approximation to \(2\) at position \(n\) is accurate to \(2^{-(n+k_n)}\), as no adjustments are made by adding 1s during this interval.

\subsection*{Precision Assumption}

Numerical computations and theoretical analysis are assumed to use sufficient precision, specifically maintaining precision levels \(P \geq n + k_n + 1\) bits, to avoid errors from truncation.

\subsection*{Algebraic Properties of \(2\)}

\(2\) is irrational and algebraic of degree 2. Roth’s theorem provides a bound on the quality of rational approximations to \(2\):
\[
\left|2 - \frac{p}{q}\right| > \frac{c}{q^2},
\]
for all integers \(p, q\) with \(q > 0\) and some constant \(c > 0\). This is fundamental to limiting the length of zero runs.

\subsection*{Positional Validity}

The conjecture applies to positions \(n \geq n_0\), where \(n_0\) is a sufficiently large integer to account for any irregular behavior in the early binary expansion.

\subsection*{Growth Rate}

The bound is asymptotic, meaning that the behavior for large \(n\) aligns with the logarithmic growth rate. The constant \(C\) encapsulates small deviations that may occur due to local variations.

\section*{Restating the Conjecture with a Positional Domain}

For \(n \geq n_0\), where \(n_0\) is sufficiently large, we hypothesize:
\[
\forall n \geq n_0, \quad k_n \leq \log_2(n) + C.
\]
This restatement clarifies the positional domain and ensures that \(C\) accounts for irregularities in the binary expansion for smaller values of \(n\).

\section*{Specific Goals for Formal Proof}

To rigorously establish this conjecture, it is necessary to:
\begin{itemize}
    \item Relate \(k_n\) directly to the rational approximation properties of \(2\).
    \item Use inequalities derived from Roth's theorem or similar results to constrain the error in rational approximations.
    \item Prove that any \(k_n > \log_2(n) + C\) would violate these constraints.
    \item Determine the explicit value of \(C\) or show that it is negligible compared to \(\log_2(n)\).
\end{itemize}

\section*{Step-by-Step Derivation of the Constant \(C\)}

\subsection*{Step 1: Roth's Theorem for Quadratic Irrationals}

Roth's theorem states that for an irrational algebraic number \(\alpha\) of degree \(d\), there exists a constant \(c > 0\) such that:
\[
\left|\alpha - \frac{p}{q}\right| > \frac{c}{q^2},
\]
for any integers \(p, q\) with \(q > 0\).

For quadratic irrationals (degree \(d = 2\)), this bound is sharp, and \(c\) depends on the specific number \(\alpha = \sqrt{2}\). However, the precise value of \(c\) is not given explicitly in Roth's theorem but can be inferred from related results.

\subsection*{Step 2: Known Bounds for \(c\)}

For \(\sqrt{2}\), the value of \(c\) can be linked to the continued fraction expansion of \(\sqrt{2}\), which provides "best" rational approximations. The continued fraction for \(\sqrt{2}\) is periodic:
\[
\sqrt{2} = [1; 2, 2, 2, \ldots],
\]
and its convergents \(p_k / q_k\) satisfy:
\[
\left|\sqrt{2} - \frac{p_k}{q_k}\right| \sim \frac{1}{q_k^2}.
\]
Explicitly, the best constant \(c\) for \(\sqrt{2}\) can be approximated from its continued fraction properties:
\[
c = \frac{1}{8}.
\]
This comes from the discriminant of the quadratic form defining \(\sqrt{2}\).

\subsection*{Step 3: Substitute \(c\) into the Formula for \(C\)}

From earlier, we derived:
\[
C = -1 - \log_2(c).
\]
Substitute \(c = \frac{1}{8} = 2^{-3/2}\):
\[
\log_2(c) = \log_2(2^{-3/2}) = -\frac{3}{2}.
\]
Thus:
\[
C = -1 - \left(-\frac{3}{2}\right) = -1 + \frac{3}{2} = \frac{1}{2}.
\]

\subsection*{Step 4: Final Expression for \(C\)}

The explicit constant \(C\) for the conjecture is:
\[
C = \frac{1}{2}.
\]

\subsection*{Step 5: Interpretation}

This means the length of a zero run in the binary expansion of \(\sqrt{2}\) at position \(n\) is bounded by:
\[
k_n \leq \log_2(n) + \frac{1}{2}.
\]

\subsection*{Step 6: Validation and Refinement}

\begin{itemize}
    \item \textbf{Empirical Validation:} Use computational tools to verify that observed zero runs fit within this bound.
    \item \textbf{Extensions:} This method can be generalized to other quadratic irrationals, where \(c\) will depend on the continued fraction discriminant.
\end{itemize}

\section*{Generalizing to Other Quadratic Irrationals}

To extend this method to other quadratic irrationals, we can rely on the properties of continued fractions and the discriminant of the quadratic equation that defines these numbers. The process involves analyzing the constants \(c\) and \(C\) in Roth's theorem, as well as the periodicity and structure of their continued fractions.

\subsection*{Step 1: Representing a Quadratic Irrational}

Any quadratic irrational \(\alpha\) can be expressed as the solution to a quadratic equation:
\[
a\alpha^2 + b\alpha + c = 0,
\]
where \(a, b, c \in \mathbb{Z}\) are coprime, and \(b^2 - 4ac > 0\) is the discriminant.

The continued fraction expansion of \(\alpha\) is periodic, which provides a direct connection to its approximations. For example, for \(\sqrt{D}\) (a square root of a positive nonsquare integer \(D\)):
\[
\sqrt{D} = [a_0; a_1, a_2, \ldots, a_m\overline{a_1, a_2, \ldots, a_m}],
\]
where the sequence \(a_1, a_2, \ldots, a_m\) repeats.

\subsection*{Step 2: Roth's Theorem and \(c\) for General Quadratic Irrationals}

For any quadratic irrational \(\alpha\):
\[
\left|\alpha - \frac{p}{q}\right| > \frac{c}{q^2},
\]
where \(c > 0\) depends on the discriminant \(\Delta = b^2 - 4ac\) of the quadratic equation defining \(\alpha\).

For \(\sqrt{D}\), where \(D > 0\):
The constant \(c\) can be approximated as:
\[
c = \frac{1}{2\sqrt{D}}.
\]
This result arises because the best rational approximations of \(\sqrt{D}\) (derived from the convergents of its continued fraction expansion) satisfy:
\[
\left|\sqrt{D} - \frac{p_k}{q_k}\right| \sim \frac{1}{2q_k^2\sqrt{D}}.
\]

\subsection*{Step 3: Deriving \(C\) for Quadratic Irrationals}

Substitute \(c = \frac{1}{2\sqrt{D}}\) into the formula for \(C\):
\[
C = -1 - \log_2(c).
\]
Simplify:
\[
\log_2(c) = \log_2\left(\frac{1}{2\sqrt{D}}\right) = -\log_2(2\sqrt{D}) = -1 - \frac{1}{2}\log_2(D).
\]
Thus:
\[
C = -1 - \left(-1 - \frac{1}{2}\log_2(D)\right) = \frac{1}{2}\log_2(D).
\]

\subsection*{Step 4: Generalized Bound for Zero Runs}

The length of a zero run in the binary expansion of a quadratic irrational \(\alpha = \sqrt{D}\) at position \(n\) is bounded by:
\[
k_n \leq \log_2(n) + \frac{1}{2}\log_2(D).
\]
Here:
\begin{itemize}
    \item \(\log_2(n)\) reflects the primary logarithmic growth of zero runs,
    \item \(\frac{1}{2}\log_2(D)\) is a constant offset that depends on the discriminant \(D\).
\end{itemize}

\subsection*{Step 5: Examples}

\textbf{Example 1: \(\sqrt{3}\)}

Discriminant: \(D = 3\),

Constant \(C\):
\[
C = \frac{1}{2}\log_2(3) \approx 0.79.
\]
The bound becomes:
\[
k_n \leq \log_2(n) + 0.79.
\]

\textbf{Example 2: \(\sqrt{5}\)}

Discriminant: \(D = 5\),

Constant \(C\):
\[
C = \frac{1}{2}\log_2(5) \approx 1.16.
\]
The bound becomes:
\[
k_n \leq \log_2(n) + 1.16.
\]

\textbf{Example 3: \(\frac{1 + \sqrt{5}}{2}\) (Golden Ratio)}

For the quadratic irrational \(\frac{1 + \sqrt{5}}{2}\), the discriminant is still \(D = 5\), so the same value of \(C\) applies:
\[
k_n \leq \log_2(n) + 1.16.
\]

\subsection*{Step 6: Implications and Further Research}

\textbf{Broader Generalization:} This method can be extended to quadratic irrationals in the form \((A + B\sqrt{D})/C\), where \(A, B, C \in \mathbb{Z}\). The discriminant will depend on the coefficients of the quadratic equation.

\textbf{Connection to Continued Fractions:} The periodic structure of the continued fraction expansion provides direct information about the quality of approximations, which affects the bound on zero runs.

\textbf{Applications to Normality:} Analyze whether the zero run bounds imply statistical properties (like normality) for quadratic irrationals.