
\section*{Abstract}
This paper presents a comprehensive analysis of consecutive zero runs in the binary expansion of $\sqrt{2}$. I investigate the conjecture that for sufficiently large position $n$, there cannot be a run of zeros longer than $\log_2(n)$. Through both Diophantine approximation theory and computational verification, I explore the mathematical structure underlying this conjecture. My analysis combines theoretical frameworks with high-precision numerical investigations, revealing fundamental constraints that support the conjecture while identifying key patterns in the distribution of zero runs. I further highlight the practical significance of these findings by detailing novel algorithmic approaches and computational methods. Rigorous error analysis and detailed scaling studies provide robust evidence for the conjecture's validity, suggesting broader implications for irrational number approximations and their applications in cryptography and computational mathematics.
