\documentclass[12pt]{article}

% Document encoding
\usepackage[utf8]{inputenc}

% Page size and margins
\usepackage[a4paper, margin=1in]{geometry}

% Math packages
\usepackage{amsmath}
\usepackage{amsthm}
\usepackage{amssymb}

% Graphics and visualization
\usepackage{graphicx}
\usepackage{tikz}
\usetikzlibrary{shapes.geometric, shapes.misc, arrows.meta, positioning, calc}

\usepackage{pgfplots}
\pgfplotsset{compat=1.18}

% Algorithm and code
\usepackage{algorithm}
\usepackage{algpseudocode}
\usepackage{listings}
\usepackage{xcolor}

% Document formatting
\usepackage{tikz}
\usepackage{float}
\usepackage{fullpage}
\usepackage{titlesec}
\usepackage{caption}

% Markdown support
\usepackage{markdown}

% References and links (load last)
\usepackage[colorlinks=true,linkcolor=blue,urlcolor=blue]{hyperref}
\usepackage{doi}

% Consistent spacing for sections
\titlespacing*{\section}{0pt}{3.5ex plus 1ex minus .2ex}{2.3ex plus .2ex}
\titlespacing*{\subsection}{0pt}{3.25ex plus 1ex minus .2ex}{1.5ex plus .2ex}

% Python code style definition
% Color selections: black, blue, brown, cyan, darkgray, gray, green, lightgray, lime, magenta, olive, orange, pink, purple, red, teal, violet, white, yellow.
\lstdefinestyle{pythonstyle}{
    language=Python,
    basicstyle=\ttfamily\footnotesize,
    keywordstyle=\color{blue},
    commentstyle=\color{olive},
    stringstyle=\color{red},
    showstringspaces=false,
    numbers=left,
    numberstyle=\tiny\color{gray},
    numbersep=10pt,
    tabsize=4,
    breaklines=true,
    breakatwhitespace=false,
    frame=single,
    captionpos=b,
    postbreak=\raisebox{0ex}[0ex][0ex]{\ensuremath{\color{red}\hookrightarrow\space}}
}
\lstset{style=pythonstyle}

% Theorem environments
\theoremstyle{plain}
\newtheorem{theorem}{Theorem}[section]
\newtheorem{lemma}[theorem]{Lemma}
\newtheorem{corollary}[theorem]{Corollary}
\newtheorem{conjecture}[theorem]{Conjecture}

% Theorem environments with different style
\theoremstyle{remark}
\newtheorem{remark}[theorem]{Remark}

% Graphic path
\graphicspath{{../Code/data/}}



% Common mathematical notation
\newcommand{\sqrttwo}{\sqrt{2}}
\newcommand{\logn}{\log_2(n)}
\newcommand{\eps}{\varepsilon}
