\subsection{Zero Runs Normality Analysis}

Building upon our previous examination of the binary expansion properties of $\sqrt{2}$, we now turn to a detailed analysis of zero run distributions. This analysis provides crucial insights into the structural patterns that emerge in the binary representation, offering a complementary perspective to the frequency analysis presented in Sections 3.1-3.9.

\subsection{Zero Run Normality Algorithm}
Building upon our previous work, the Zero Run Normality Algorithm analyzes the distribution of zero runs to evaluate normality properties.

\subsubsection{Algorithm Workflow}
We summarize the algorithm as pseudocode for clarity:

\begin{algorithm}[H]
\caption{Zero Run Normality Analysis}
\begin{algorithmic}[1]
\State \textbf{Input:} Binary expansion of $\sqrt{2}$ up to $N$ digits
\State Compute the frequency of zero runs of each length $k$
\For{each run length $k$}
    \State Compute distribution $P(k)$
    \State Compare with theoretical prediction $P_{\text{theoretical}}(k) = 2^{-(k+1)}$
\EndFor
\State Perform statistical tests (e.g., Kolmogorov-Smirnov)
\State \textbf{Output:} Deviations, entropy values, and normality test results
\end{algorithmic}
\end{algorithm}

\subsubsection{Flowchart}
A high-level flowchart (Figure \ref{fig:normality_flowchart}) is provided for better understanding:

\begin{figure}[H]
    \centering
    \includegraphics[width=0.7\textwidth]{flowchart_Zero_Run_Normality_Analysis.png}
    \caption{Flowchart of the Zero Run Normality Analysis Algorithm.}
    \label{fig:normality_flowchart}
\end{figure}\subsubsection{Motivation and Connection to Previous Analysis}
The study of zero runs directly extends our understanding of digit patterns discussed in Section 3.3 by examining consecutive sequences of zeros rather than individual digit frequencies. This approach reveals deeper structural properties that are not immediately apparent from simple frequency analysis:

\begin{itemize}
    \item While Section 3.4 examined individual digit distributions, zero run analysis captures higher-order correlations between digits
    \item The methods developed in Section 3.7 for pattern detection are now expanded to identify longer-range dependencies
    \item The statistical framework from Section 3.8 is enhanced to handle sequence-based analysis
\end{itemize}

\subsubsection{Methodological Framework}
Our analysis framework extends the statistical approaches introduced in Section 3.5 with five specialized components:

\begin{enumerate}
    \item \textbf{Block Analysis:} Extending the local analysis methods from Section 3.6, we define:
    \begin{equation}
        B_n(k) = \text{block of } k \text{ bits starting at position } n
    \end{equation}
    
    \textbf{Local Density Function:} 
    \begin{equation}
        \rho(n,k) = \frac{\text{number of zeros in }B_n(k)}{k}
    \end{equation}

    \item \textbf{Distribution Analysis:} Building on the distributional properties established in Section 3.2:
    \begin{equation}
        P(l) = \frac{\text{frequency of zero runs of length }l}{\text{total number of zero runs}}
    \end{equation}
    
    Theoretical prediction for normal numbers:
    \begin{equation}
        P_{\text{theoretical}}(l) = 2^{-(l+1)}
    \end{equation}

    \item \textbf{Entropy Measures:} Complementing the complexity measures from Section 3.8:
    \begin{equation}
        H_B(k) = -\sum_{i} p_i(k) \log_2 p_i(k)
    \end{equation}
    \begin{equation}
        H_R = -\sum_{l} P(l) \log_2 P(l)
    \end{equation}

    \item \textbf{Discrepancy Analysis:} Extending the error bounds from Section 3.9:
    \begin{equation}
        D_N = \sup_{0 \leq x \leq 1} |F_N(x) - x|
    \end{equation}

    \item \textbf{Pattern Structure Analysis:} Building on the structural analysis from Section 3.7:
    \begin{equation}
        C(r) = \frac{1}{N-r} \sum_{i=1}^{N-r} z_i z_{i+r}
    \end{equation}
\end{enumerate}

\section{Empirical Normality Analysis}
The Zero Run Normality Analysis algorithm was applied to the binary expansion of $\sqrt{2}$ to analyze zero run distributions. The algorithm leverages GPU acceleration for efficient computation of large-scale expansions up to $10^6$ digits.

\subsection{Connection to Normality Properties}
Our analysis provides strong empirical evidence for the normality conjecture:
\begin{itemize}
    \item The observed zero run distributions exhibit geometric decay with rate $2^{-(k+1)}$ for runs of length $k$
    \item Statistical testing using the Kolmogorov-Smirnov test yielded p-values consistently above the $\alpha = 0.01$ significance level
    \item The maximum observed discrepancy remained bounded by $O(\log n/n)$ across all scales
\end{itemize}

\subsection{Implementation Requirements}
The analysis framework maintains rigorous computational standards:
\begin{itemize}
    \item High-precision computation using arbitrary-precision arithmetic with $10^6$ digits
    \item GPU-accelerated binary expansion generation and analysis
    \item Multi-scale analysis spanning block sizes from $2^1$ to $2^{20}$ bits
    \item Statistical significance testing at $\alpha = 0.01$ level
\end{itemize}

\subsection{Results and Interpretation}
Key findings from our empirical analysis:

\begin{figure}[htbp]
    \centering
    \includegraphics[width=\textwidth]{normality_analysis_1000000.png}
    \caption{Normality analysis of $\sqrt{2}$ binary expansion 1,000,000 digit evaluation showing (a) zero run distribution, (b) scale-dependent entropy, (c) discrepancy bounds, and (d) Q-Q plot against theoretical predictions.}
    \label{fig:normality_analysis}
\end{figure}

\begin{itemize}
    \item Zero run distributions closely follow theoretical predictions with deviations bounded by $O(\log n/n)$
    \item Scale-dependent entropy analysis reveals no significant deviation from expected values for normal numbers
    \item Maximum observed discrepancy of $\approx 7.6 \times 10^{-4}$ for $n = 10^6$ digits
    \item Kolmogorov-Smirnov test p-value of $3.89 \times 10^{-2}$ supports normality hypothesis
\end{itemize}

\newpage
\subsection{Future Directions}
This analysis suggests several promising research directions:
\begin{itemize}
    \item Extension to higher-order pattern analysis beyond simple zero runs
    \item Investigation of connections between binary normality and continued fraction expansions
    \item Development of more efficient algorithms for testing normality in quadratic irrationals
    \item Analysis of cross-scale correlations in the binary expansion
\end{itemize}