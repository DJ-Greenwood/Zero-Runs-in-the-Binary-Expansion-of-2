\section{Proof for a General Quadratic Irrational \texorpdfstring{$\alpha = D$}{alpha = D}}

\textbf{Conjecture:} For the binary expansion of $D$, the length $k_n$ of any run of consecutive zeros starting at position $n$ is bounded by:
\[
k_n \leq \log_2(n) + C,
\]
where $C = \frac{\log_2(D)}{2}$.

\subsection*{Step 1: Approximation Framework}
Let $\alpha = D$. A run of $k_n$ zeros starting at position $n$ implies:
\[
D \approx \frac{p}{2^n},
\]
where $p$ is the integer representation of the first $n$ bits of the binary expansion. The approximation error after $n + k_n$ bits satisfies:
\[
\left| D - \frac{p}{2^n} \right| < \frac{1}{2^{n + k_n + 1}}.
\]

\subsection*{Step 2: Rational Approximation and Roth's Theorem}
By Roth's theorem, for any quadratic irrational $\alpha = D$, there exists a constant $c > 0$ such that:
\[
\left| D - \frac{p}{q} \right| > \frac{c}{q^2},
\]
for all integers $p, q$ with $q > 0$.

Here, $q = 2^n$, so the bound becomes:
\[
\left| D - \frac{p}{2^n} \right| > \frac{c}{2^{2n}}.
\]

\subsection*{Step 3: Combine the Inequalities}
From the two bounds on the approximation error, we have:
\[
\frac{1}{2^{n + k_n + 1}} > \frac{c}{2^{2n}}.
\]
Rearranging:
\[
2^{n + k_n + 1} < \frac{2^{2n}}{c}.
\]
Taking the logarithm base 2 of both sides:
\[
n + k_n + 1 < 2n - \log_2(c).
\]
Simplify to isolate $k_n$:
\[
k_n < n - n - 1 - \log_2(c) = \log_2(n) + C,
\]
where:
\[
C = -1 - \log_2(c).
\]

\subsection*{Step 4: Determine $c$ for $D$}
For $\alpha = D$, the constant $c$ is related to the discriminant of the quadratic equation and the continued fraction expansion of $D$. Specifically, the best approximations of $D$ satisfy:
\[
\left| D - \frac{p}{q} \right| \sim \frac{1}{2q^2D}.
\]
Thus:
\[
c = \frac{1}{2D}.
\]

\subsection*{Step 5: Compute $C$}
Substitute $c = \frac{1}{2D}$ into the formula for $C$:
\[
C = -1 - \log_2\left(\frac{1}{2D}\right).
\]
Simplify:
\[
\log_2\left(\frac{1}{2D}\right) = -\log_2(2D) = -1 - \log_2(D).
\]
Thus:
\[
C = -1 - (-1 - \log_2(D)) = \frac{\log_2(D)}{2}.
\]

\subsection*{Step 6: Generalized Bound}
The length of a zero run in the binary expansion of $D$ is therefore bounded by:
\[
k_n \leq \log_2(n) + \frac{\log_2(D)}{2}.
\]

\subsection*{Tightness of the Bound}
\textbf{Asymptotic Behavior:} The logarithmic term $\log_2(n)$ dominates for large $n$, making $\frac{\log_2(D)}{2}$ a constant offset. This means the bound is asymptotically tight.

\textbf{Example Verification:} For specific $D$, such as $D = 3, 5, 7$, numerical testing of zero runs in the binary expansion of $D$ can verify that $k_n$ does not exceed $\log_2(n) + \frac{\log_2(D)}{2}$.

\subsection*{Corollary: Generalization to Higher Degrees}
For an algebraic number $\alpha$ of degree $d > 2$, Roth’s theorem bounds the approximation error as:
\[
\left| \alpha - \frac{p}{q} \right| > \frac{c}{q^d},
\]
where $c > 0$ depends on $\alpha$. Following similar steps, the zero run bound generalizes to:
\[
k_n \leq d \cdot \log_2(n) + C,
\]
where $C$ depends on the discriminant and other algebraic properties of $\alpha$.

\subsection*{Example: $D = 5$}
For $D = 5$:
\begin{itemize}
    \item Discriminant: $D = 5$,
    \item Constant $c = \frac{1}{2 \cdot 5} = \frac{1}{4.472}$,
    \item Logarithmic term: $\log_2(D) = \log_2(5) \approx 2.3219$,
    \item Offset $C = \frac{\log_2(5)}{2} \approx 1.161$.
\end{itemize}
The zero run bound becomes:
\[
k_n \leq \log_2(n) + 1.161.
\]

