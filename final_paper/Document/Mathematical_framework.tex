\section{Mathematical Framework}

\subsection{Representation of Zero Runs}
The binary expansion of $\sqrt{2}$ is an infinite sequence of 0s and 1s that, when interpreted as a binary number, equals $\sqrt{2}$. In this expansion, we occasionally encounter consecutive sequences of zeros, which we call "zero runs." To analyze these patterns mathematically, we need a precise way to represent them.

Consider a specific position $n$ in this binary expansion where we observe a run of $k$ consecutive zeros. We can represent this portion of $\sqrt{2}$ as:
\[
\sqrt{2} = \frac{p}{2^n} + \frac{q}{2^{n+k}}
\]
where:
\begin{itemize}
    \item $p$ represents the numerical value obtained by interpreting the first $n$ binary digits as a binary number.
    \item $q$ represents the numerical value of all digits that appear after the zero run (after position $n+k$).
    \item The $k$ zeros between positions $n$ and $n+k$ are implicitly represented by the difference in exponents between the denominators.
\end{itemize}

\subsection{Key Equations}
Starting from the representation:
\[
\sqrt{2} = \frac{p}{2^n} + \frac{q}{2^{n+k}}
\]
we perform the following transformations to simplify and reveal important properties.

1. \textbf{Eliminate Fractions}: Multiply both sides by $2^n$:
\[
2^n \sqrt{2} = p + \frac{q}{2^k}.
\]

2. \textbf{Remove the Irrational Term}: Square both sides:
\[
(2^n \sqrt{2})^2 = \left(p + \frac{q}{2^k}\right)^2.
\]

3. \textbf{Expand and Simplify}: Using the binomial expansion:
\[
2^{2n} \cdot 2 = p^2 + \frac{2pq}{2^k} + \frac{q^2}{2^{2k}}.
\]

4. \textbf{Isolate Terms}: Rearrange to group powers of 2:
\[
2^{2n+1} - p^2 = \frac{2pq}{2^k} + \frac{q^2}{2^{2k}}.
\]

5. \textbf{Work with Integers}: Multiply through by $2^{2k}$ to avoid fractions:
\[
2^{2n+2k+1} - p^2 \cdot 2^{2k} = 2pq \cdot 2^k + q^2.
\]

This final form is expressed entirely in integers, allowing us to analyze the relationships between $n$, $k$, $p$, and $q$.

\subsection{Fundamental Lemmas}
The following lemmas provide bounds on the behavior of zero runs in the binary expansion of $\sqrt{2}$, connecting them to classical results in number theory.

\textbf{Lemma 1: Rational Approximation Bound.}  
For any position $n$ and run length $k$, the error in approximating $\sqrt{2}$ by $\frac{p}{2^n}$ satisfies:
\[
\left|\sqrt{2} - \frac{p}{2^n}\right| > \frac{c}{2^{2n}},
\]
where $c > 0$ is a constant.

\textit{Intuitive Explanation:}  
This lemma ensures that the binary approximation of $\sqrt{2}$ cannot be too precise. Since $\sqrt{2}$ is algebraic of degree 2, Roth's theorem limits the quality of rational approximations, and this bound reflects that limit.

\textit{Proof (Simplified):}  
1. Assume the contrary: that the error is smaller than $\frac{c}{2^{2n}}$ for infinitely many $n$.
2. Such an error would contradict Roth's theorem, which states that algebraic numbers cannot be approximated by rationals with error smaller than $\frac{1}{2^{(2+\delta)n}}$ for any $\delta > 0$. 
3. Therefore, the stated bound must hold. \qed

\textbf{Lemma 2: Zero Run Length Bound.}  
For a zero run of length $k$ starting at position $n$:
\[
k < 2 \log_2(n) + O(1).
\]

\textit{Intuitive Explanation:}  
A long zero run implies using the same rational approximation for many consecutive bits, which increases the approximation error. This lemma limits the length of such runs relative to their starting position.

\textit{Proof (Simplified):}  
1. The error due to a zero run of length $k$ is at least:
\[
\frac{1}{2^{n+k+1}},
\]
because the next bit after the zero run must be 1.

2. By Lemma 1, the error must also satisfy:
\[
\left|\sqrt{2} - \frac{p}{2^n}\right| < \frac{c}{2^{2n}}.
\]

3. Combining these bounds:
\[
\frac{1}{2^{n+k+1}} < \frac{c}{2^{2n}}.
\]

4. Taking logarithms:
\[
k < 2 \log_2(n) + O(1).
\] \qed

\subsection{Connecting Lemmas to Zero Run Patterns}
These lemmas highlight the relationship between:
\begin{enumerate}
    \item Diophantine approximation theory (e.g., Roth's theorem),
    \item Rational approximations of $\sqrt{2}$, and
    \item The structural constraints on zero runs in the binary expansion.
\end{enumerate}

The logarithmic bound on zero run lengths indicates that while arbitrarily long runs can occur, their likelihood decreases significantly at higher positions in the binary expansion. This provides a clear measure of the complexity inherent in the binary representation of $\sqrt{2}$.
