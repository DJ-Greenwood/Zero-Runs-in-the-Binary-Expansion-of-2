\section{Mathematical Framework}

\subsection{Representation of Zero Runs}
The binary expansion of $\sqrt{2}$ is an infinite sequence of 0s and 1s that, when interpreted as a binary number, equals $\sqrt{2}$. In this expansion, we occasionally encounter consecutive sequences of zeros, which we call ``zero runs.'' To analyze these patterns mathematically, we need a precise way to represent them.

Consider a specific position $n$ in this binary expansion where we observe a run of $k$ consecutive zeros. We can represent this portion of $\sqrt{2}$ as:
\[
\sqrt{2} = \frac{p}{2^n} + \frac{q}{2^{n+k}}
\]
where:
\begin{itemize}
    \item $p$ represents the numerical value obtained by interpreting the first $n$ binary digits as a binary number.
    \item $q$ represents the numerical value of all digits that appear after the zero run (after position $n+k$).
    \item The $k$ zeros between positions $n$ and $n+k$ are implicitly represented by the difference in exponents between the denominators.
\end{itemize}

\subsection{Key Equations}
Our analysis begins with the representation developed above. Through a series of algebraic transformations, we convert this representation into a form that reveals important properties of these zero runs.

Starting with our representation:
\[
\sqrt{2} = \frac{p}{2^n} + \frac{q}{2^{n+k}}
\]

To eliminate fractions and simplify our analysis, we multiply both sides by $2^n$:
\[
2^n \sqrt{2} = p + \frac{q}{2^k}
\]

Since we're working with $\sqrt{2}$, squaring both sides allows us to eliminate the irrational number:
\[
(2^n \sqrt{2})^2 = \left(p + \frac{q}{2^k}\right)^2
\]

Expanding the right side using the square of a binomial and simplifying the left side:
\[
2^{2n} \cdot 2 = p^2 + \frac{2pq}{2^k} + \frac{q^2}{2^{2k}}
\]

Rearranging to isolate terms with different powers of 2:
\[
2^{2n+1} - p^2 = \frac{2pq}{2^k} + \frac{q^2}{2^{2k}}
\]

To work with integer values, we multiply all terms by $2^{2k}$:
\[
2^{2n+2k+1} - p^2 \cdot 2^{2k} = 2pq \cdot 2^k + q^2
\]

This final equation, expressed entirely in integers, provides a powerful tool for analyzing the relationships between $n$, $k$, $p$, and $q$, ultimately allowing us to establish constraints on the possible lengths of zero runs.

\subsection{Fundamental Lemmas}
The behavior of zero runs in the binary expansion of $\sqrt{2}$ is governed by deep properties from number theory. The following lemmas connect classical results about Diophantine approximation to specific properties of binary expansions.

\textbf{Lemma 1: Rational Approximation Bound.} 
This lemma establishes a fundamental limit on how well $\sqrt{2}$ can be approximated by rational numbers of the form $\frac{p}{2^n}$. Specifically, for any position $n$ and run length $k$, if $\frac{p}{2^n}$ approximates $\sqrt{2}$, then:
\[
\left|\sqrt{2} - \frac{p}{2^n}\right| > \frac{c}{2^{2n}}
\]
for some constant $c > 0$.

\textit{Intuition:} This bound tells us that when we truncate the binary expansion of $\sqrt{2}$ at position $n$ (getting a rational approximation $\frac{p}{2^n}$), the error can't be smaller than $\frac{c}{2^{2n}}$. The exponent 2 appears because $\sqrt{2}$ is algebraic of degree 2.

\textit{Proof.} We proceed by contradiction. Assume no such $c$ exists. Then for any $\epsilon > 0$, there exist infinitely many $n$ with:
\[
\left|\sqrt{2} - \frac{p}{2^n}\right| < \frac{\epsilon}{2^{2n}}
\]
This would provide approximations violating Roth's theorem, which states that algebraic numbers of degree 2 cannot be approximated by rationals with error better than $\frac{1}{2^{(2+\delta)n}}$ for any $\delta > 0$. \qed

\textbf{Lemma 2: Zero Run Length Bound.} 
This lemma translates the approximation bound into a concrete limit on zero run lengths. For a zero run of length $k$ starting at position $n$:
\[
k < 2 \log_2(n) + O(1)
\]

\textit{Intuition:} A long run of zeros means we're using the same rational approximation for many bits. This lemma shows that such runs cannot be too long relative to their position in the expansion.

\textit{Proof.} The key insight is that if we have a run of $k$ zeros starting at position $n$, then:
\begin{itemize}
    \item The approximation error must be at least $\frac{1}{2^{n+k+1}}$ (since the next bit is 1)
    \item But by Lemma 1, the error is also less than $\frac{c}{2^{2n}}$
\end{itemize}
Therefore:
\[
\frac{1}{2^{n+k+1}} < \left|\sqrt{2} - \frac{p}{2^n}\right| < \frac{c}{2^{2n}}
\]
Taking logarithms and solving for $k$ yields the result. \qed

These lemmas connect three different perspectives:
\begin{enumerate}
    \item The abstract theory of Diophantine approximation (Roth's theorem)
    \item Rational approximations of $\sqrt{2}$
    \item The concrete structure of zero runs in the binary expansion
\end{enumerate}

The logarithmic bound on zero run lengths shows that while arbitrarily long runs of zeros can occur, they become increasingly rare as we progress further in the expansion. This provides a quantitative measure of the complexity in the binary expansion of $\sqrt{2}$.
