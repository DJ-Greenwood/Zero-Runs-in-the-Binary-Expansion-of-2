\subsection{Zero Runs Normality Analysis}

Building upon our previous examination of the binary expansion properties of $\sqrt{2}$, we now turn to a detailed analysis of zero run distributions. This analysis provides crucial insights into the structural patterns that emerge in the binary representation, offering a complementary perspective to the frequency analysis presented in Sections 3.1-3.9.

\subsubsection{Motivation and Connection to Previous Analysis}
The study of zero runs directly extends our understanding of digit patterns discussed in Section 3.3 by examining consecutive sequences of zeros rather than individual digit frequencies. This approach reveals deeper structural properties that are not immediately apparent from simple frequency analysis:

\begin{itemize}
    \item While Section 3.4 examined individual digit distributions, zero run analysis captures higher-order correlations between digits
    \item The methods developed in Section 3.7 for pattern detection are now expanded to identify longer-range dependencies
    \item The statistical framework from Section 3.8 is enhanced to handle sequence-based analysis
\end{itemize}

\subsubsection{Methodological Framework}
Our analysis framework extends the statistical approaches introduced in Section 3.5 with five specialized components:

\begin{enumerate}
    \item \textbf{Block Analysis:} Extending the local analysis methods from Section 3.6, we define:
    \begin{equation}
        B_n(k) = \text{block of } k \text{ bits starting at position } n
    \end{equation}
    
    \textbf{Local Density Function:} 
    \begin{equation}
        \rho(n,k) = \frac{\text{number of zeros in }B_n(k)}{k}
    \end{equation}

    \item \textbf{Distribution Analysis:} Building on the distributional properties established in Section 3.2:
    \begin{equation}
        P(l) = \frac{\text{frequency of zero runs of length }l}{\text{total number of zero runs}}
    \end{equation}
    
    Theoretical prediction for normal numbers:
    \begin{equation}
        P_{\text{theoretical}}(l) = 2^{-(l+1)}
    \end{equation}

    \item \textbf{Entropy Measures:} Complementing the complexity measures from Section 3.8:
    \begin{equation}
        H_B(k) = -\sum_{i} p_i(k) \log_2 p_i(k)
    \end{equation}
    \begin{equation}
        H_R = -\sum_{l} P(l) \log_2 P(l)
    \end{equation}

    \item \textbf{Discrepancy Analysis:} Extending the error bounds from Section 3.9:
    \begin{equation}
        D_N = \sup_{0 \leq x \leq 1} |F_N(x) - x|
    \end{equation}

    \item \textbf{Pattern Structure Analysis:} Building on the structural analysis from Section 3.7:
    \begin{equation}
        C(r) = \frac{1}{N-r} \sum_{i=1}^{N-r} z_i z_{i+r}
    \end{equation}
\end{enumerate}

\subsection{Empirical Normality Analysis}

The \texttt{$Zero\_Run\_Normality\_Analysis$} algorithm was applied to the binary expansion of $\sqrt{2}$ to analyze zero run distributions. The results were compared against theoretical predictions for normal numbers, focusing on the following aspects:


\subsubsection{Connection to Normality Properties}
This analysis provides crucial evidence for the normality conjecture discussed in Section 3.1:

\begin{itemize}
    \item The observed zero run distributions closely match theoretical predictions for normal numbers
    \item Entropy calculations suggest the absence of algorithmic compressibility
    \item Discrepancy measures remain bounded in accordance with normality criteria
\end{itemize}

\subsubsection{Implementation Requirements}
To maintain consistency with the precision standards established in Section 3.2:

\begin{itemize}
    \item Computation requires minimum $10^6$ binary digits of $\sqrt{2}$
    \item Statistical testing at $\alpha = 0.01$ level
    \item Analysis spans scales from $2^1$ to $2^{20}$ bits
\end{itemize}

\subsubsection{Results and Interpretation}
These findings complement our earlier results:

\begin{itemize}
    \item Zero run distributions exhibit geometric decay with $O(\log n/n)$ bounded deviations
    \item Block entropy calculations reveal scale-dependent structure
    \item Results support the conjectured $\log_2(n)$ bound from Section 3.4
\end{itemize}

\subsubsection{Future Directions}
This analysis suggests several promising extensions of the work presented in Section 3:

\begin{itemize}
    \item Investigation of higher-order run patterns
    \item Connection to continued fraction expansions
    \item Application to other quadratic irrationals
\end{itemize}