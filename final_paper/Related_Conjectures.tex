\section{Related Conjectures}

\subsection{Binary Normality}
The distribution of zeros in $\sqrt{2}$ relates to the broader question of normality in number theory. A number is considered normal in base 2 if every possible finite sequence of digits appears with the expected limiting frequency. This property has profound implications for the randomness and structure of the number's binary expansion.

\textbf{Theorem 1 (Conditional Normality):} If the $\log_2(n)$ bound holds, then the frequency of zero runs of length $k$ in $\sqrt{2}$ is bounded above by $2^{-k}(1 + o(1))$. This result connects our local structural analysis to global statistical properties of the expansion, suggesting that $\sqrt{2}$ exhibits behavior characteristic of normal numbers.

\subsection{Generalization to Algebraic Numbers}
Evidence suggests similar bounds may hold for other algebraic numbers, pointing to a deeper connection between algebraic degree and binary expansion properties. This generalization would establish a fundamental relationship between a number's algebraic complexity and the structure of its binary representation.

\textbf{Conjecture 1 (Generalized Run Length):} For any algebraic number $\alpha$ of degree $d$, runs of zeros in its binary expansion are bounded by $d \log_2(n)$ at position $n$. This conjecture proposes that the algebraic degree directly influences the maximum possible length of consecutive zero runs, providing a quantitative measure of how algebraic complexity constrains digit patterns.

\textbf{Theorem 2 (Zero Run Length Bound):} Let $n$ be a position in the binary expansion of $\sqrt{2}$, and let $k$ be the length of a run of zeros starting at position $n$. Define:
\begin{itemize}
    \item $p$ as the value of the first $n$ binary digits, representing the initial segment of the expansion.
    \item $q$ as the value of the digits after position $n+k$, capturing the remainder of the expansion.
    \item $c$ as a positive constant from Roth's theorem, which provides fundamental limits on rational approximation.
\end{itemize}


Then the following statements form a contradiction when $k > \log_2(n)$:
\begin{enumerate}
    \item By definition of $k$ zeros at position $n$: 
    \[
    \left|\sqrt{2} - \left(\frac{p}{2^n} + \frac{q}{2^{n+k}}\right)\right| < \frac{1}{2^{n+k+1}}
    \]
    \item From Roth’s theorem (Lemma 1): 
    \[
    \left|\sqrt{2} - \frac{p}{2^n}\right| > \frac{c}{2^{2n}}
    \]
    \item From the fundamental inequality:
    \[
    2^{2n+2k+1} - p^2 \cdot 2^{2k} \leq 2pq \cdot 2^k + q^2
    \]
    \item From binary representation constraints:
    \[
    q < 2^n
    \]
    \item From geometric constraints:
    \[
    q > 2^{(n+k-1)/2}
    \]
\end{enumerate}

\textit{Proof:} Proceeding by contradiction, assume $k > \log_2(n)$:
\begin{enumerate}
    \item From constraint (5): 
    \[
    q > 2^{(n + \log_2(n) - 1)/2}
    \]
    \item From constraint (4): 
    \[
    2^{(n + \log_2(n) - 1)/2} < 2^n
    \]
    \item This implies:
    \[
    n + \log_2(n) - 1 < 2n
    \]
    \item Simplifying:
    \[
    \log_2(n) < n + 1
    \]
    \item However, when $k > \log_2(n)$, inequalities (3) and (5) force:
    \[
    q > 2^n
    \]
    \item This directly contradicts (4).
\end{enumerate}

Therefore, $k \leq \log_2(n)$ for sufficiently large $n$.

\textbf{Remark 1:} The key insight of this proof comes from combining geometric constraints derived
from our circle-square diagram with binary representation requirements and Roth’s theorem.
These create a fundamental incompatibility when $k > \log_2(n)$. This approach provides a
new geometric perspective on the relationship between continued fraction approximations and
binary expansions.

\textbf{Corollary 1:} The bound $k \leq \log_2(n)$ is tight in the sense that there exist positions where
the run length approaches $\log_2(n)$.

\subsection{Future Directions}
Several promising directions for future research include:
\begin{itemize}
    \item Establishing rigorous bounds on constraint incompatibility by developing new techniques in Diophantine approximation theory.
    \item Investigating the relationship between $n$ and minimum possible discrepancies to understand the optimal approximation rates.
    \item Analyzing the behavior of $q$ as a function of $k$ for fixed $n$ to reveal finer structural properties of the expansion.
    \item Exploring connections to Diophantine approximation theory, particularly how the binary expansion relates to classical approximation bounds.
\end{itemize}
